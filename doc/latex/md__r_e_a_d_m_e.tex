D\+O\+C\+U\+M\+E\+N\+TO D\+E\+L\+LE S\+P\+E\+C\+I\+F\+I\+C\+HE DI \char`\"{}i\+Fermi\char`\"{}

Scopo

L\textquotesingle{}applicazione è mirata agli utenti android della nostra scuola e l\textquotesingle{}obiettivo sarebbe di rendere più semplice per loro l\textquotesingle{}uso del nostro sito della scuola. Nell\textquotesingle{}applicazione saranno fornite le funzionalità più usate e più utili del sito, in modo da che l\textquotesingle{}utente non senta più il bisogno di andare direttamente sul sito web.

Parti del progetto

-\/agenda\+: è l\textquotesingle{}equivalente del calendario giornaliero della scuola -\/calendario\+: contiene un calendario interattivo con tutti gli eventi scritti sul calendario scolastico -\/news\+: pagina principale dell\textquotesingle{}app, contiene le news -\/orario\+: equivalente alla sezione orario definitivo -\/settings\+: impostazioni dell\textquotesingle{}applicazione -\/tutorial\+: è un piccolo tutorial che parte al primo avvio in modo da spiegare all\textquotesingle{}utente come funziona questa applicazione

Comunicazione tra parti

Tramite un file model.\+java, le varie parti dell\textquotesingle{}app comunicano tra di loro(ad esempio le varie view potranno leggere le impostazioni dal model)

Divisione delle parti

-\/Castagna\+: news, orario + model -\/Sahni\+: agenda, calendario + collegamento a registro e moodle -\/Cattani\+: settings, tutorial 